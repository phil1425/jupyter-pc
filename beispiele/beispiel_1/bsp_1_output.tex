\documentclass{article}
    \usepackage[margin=0.7in]{geometry}
    \usepackage[parfill]{parskip}
    \usepackage[utf8]{inputenc}
    \usepackage{amsmath,amssymb,amsfonts,amsthm}

\begin{document}
\section*{Bestimmung des Molaren Extinktionskoeffizienten}


            \begin{table}
                \caption{Werte zur Berechnung von \epsilon}
                \centering
                \begin{tabular}{l|l|l}
            		Intensität & Extinktion & Konzentration \\
			$7.14 \pm 0.10$ & $-0.141 \pm 0.019$ & $0.00990 \pm 0.00010$ \\
			$6.96 \pm 0.10$ & $-0.168 \pm 0.019$ & $0.00980 \pm 0.00010$ \\
			$7.01 \pm 0.10$ & $-0.160 \pm 0.019$ & $0.00950 \pm 0.00010$ \\
			$6.71 \pm 0.10$ & $-0.204 \pm 0.019$ & $0.00900 \pm 0.00009$ \\
			$6.71 \pm 0.10$ & $-0.204 \pm 0.019$ & $0.00800 \pm 0.00008$ \\
			$5.62 \pm 0.10$ & $-0.382 \pm 0.022$ & $0.00500 \pm 0.00005$ \\

                \end{tabular}
            \end{table}
          

\begin{figure}
  \includegraphics{'grafik_1.pdf'}
  \caption{Bestimmung der Ausgleichsgeraden}
\end{figure}

Der Koeffizient ist 22.9 \pm 2.6

\end{document}